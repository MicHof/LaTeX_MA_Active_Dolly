% Sample file on how to use subfiles.
\documentclass[ExampleMasters.tex]{subfiles}

\begin{document}
\clearpage


\chapter{Hardware Setup}
\label{chap:hardware_setup}
\section{Utilized dolly system}
\label{sec:dolly_system}

Short overview for the dolly, including: 

\begin{itemize}
\item mechanical properties in short (weight, turning radius, max. tonnage)
\item description of function (brake, steer, countersteer, lock at highspeed)
\item difference low $<=>$ high speed
\item control system by VSE, diagnosis, display connection with truck
\item system overview picture/schematics
\end{itemize}
\section{Real-Time Environment}
\label{sec:realtime_environment}

\begin{itemize}
	\item mechanical properties of box  (dimension, currents, mounting points in dolly)
	\item computational power/limitations
	\item explain interfaces with truck/dolly (abstract)
	\item explain technical realisation of HW interface (ZIF)
	\item explain rapid-prototyping
	\item robustness
	\item programming with software ref to \ref{sec:matlab}
	\item runtime interface ref to \ref{sec:control_desk}
\end{itemize}

\section{Interfaces with Dolly}
\label{sec:interface_with_dolly}

\begin{itemize}
	\item CANbus with AngleSensor
	\item Vehicle CAN
	\item physial interface =$>$ diagnosis outlet
\end{itemize}

\section{Measurment Setup}
\label{sec:measurement_setup}
\subsection{On-board sensors}
\begin{itemize}
	\item king pin angle
	\item steering angle 
	\item speed
\end{itemize}
\subsection{Inertial measurement unit}
\label{sec:IMU}
To determine the processing delays in the control chain (refer to chapter \ref{chap:processing_time_delay}) as well as logging implementation for verification and analyses purposes a number of inertial measurement units (IMU) where utilizied throughout this thesis' work. The system at hand combined a gyroscope (L3GD20H), and an accelero- and magnetometer (LSM303D) into an IMU put on one circuit board\cite{IMU_homepage_shop}. This one-chip solution allowed for a convenient access to the sensor measurings, as the sensor outputs could be received via Inter-Integrated Circuit-protocol (I$^{2}$C)  which eliminates the need for transducer. Furthermore a high-pass filter is integrated into the IMU's accelerometer, which allows for easier compensation of the immanent drift. 

These units supply the measurings for three axes each at a maximum frequency of 1600Hz for the accelerometer and 757.6Hz for the gyroscope. \cite{accelerometer_datasheet}\cite{gyrometer_datasheet}
\end{document}
