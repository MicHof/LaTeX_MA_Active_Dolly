% Sample file on how to use subfiles.
\documentclass[ExampleMasters.tex]{subfiles}

\begin{document}
\clearpage
\chapter{Overview}
\label{chap:overview}
\section{Ongoing research}
\begin{itemize}
	\item LVC advantages/disadvantages (Baltin)
	\item safety aspects
	\item Environmental aspects
	\item development of saftey systems
\end{itemize}  \\


\label{sec:ongoing_research}
\section{Legal Situation}
\label{sec:legal_situation}
In Europe the permitted maximum length of a road train is 18.75 m and the maximum weight is 44 tonnes. But it is possible for countries to make exeptions from that rule.\cite{96/53/EC}  For example in Sweden and Finland road trains can be up to 25.25 m long with a  maximum weight of 60 tonnes.\cite{Vaegverket}
Table \ref{tab:LCV_in_different_countries} shows the maximum length and weight of LCV in different countries.

\begin{table}[h]
	\centering
	\caption{LCV in different countries\cite{Vaegverket}\cite{LCV_Australia}\Cite{LCV_USA}\cite{LCV_Canada}\Cite{LCV_Mexico}}
	\label{tab:LCV_in_different_countries}
	\begin{tabular}{l|l|l|}
		Country   & Max. Length [m] & Max. Weight [t] \\ \hline
		Sweden    &       25.25      &       60.00      \\
		Finland   &            25.25 &         60.00    \\
		Australia &      53.50       &           132.00  \\
		USA(trailers without truck)&      26.07       &    59.86        \\
		Canada & 36.88 & 63.50 \\
		Mexico & 31.00 & 75.50 \\
	\end{tabular} \\
\end{table}
\section{Market overview for existing solutions}

\label{sec:market_overview}


\end{document}
