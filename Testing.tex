% Sample file on how to use subfiles.
\documentclass[ExampleMasters.tex]{subfiles}

\begin{document}
\clearpage
\chapter{Verification and validation of steering  (2-3 Seiten)}
\label{chap:testing}

\section{Overview}
In different states of the project different kind of tests were performed. As a first step bench tests were done to verify the developed software. Following that, several tests on the actual dolly, with the dolly in standstill and no trailers connected, were made.

\section{Bench-Testing}
\label{sec:bench-testing}
\subsection{ECU-setup}
\subsection{VTM maneuver verification}
\begin{itemize}
	\item include simulation results for maneuver of controller being run inside VTM
	\item perhaps include some picture of the video-output, with the actuated truck
	\item plots for maneuver (steering input, speed over time)
	
\end{itemize}
\subsection{CAN verification}
\begin{itemize}
	\item check output from ASF with arduino
	\item check ECU CANs, make sure they differ in output
	
\end{itemize}

\subsection{Fault detection system verification}
\label{sec:fault_detect_test}

\begin{itemize}
	\item send faulty inputs
	\item show plot of faulty input and system reaction
	
\end{itemize}
\section{Processing time evaluation}
\label{chap:processing_time_delay}
\subsection{Background}
The desired solution is supposed to operate at any speed. For high speeds a quick processing and transmission time is required to ensure prompt and realtime intervention of the control system based on the measured input signals. If the delays induced by the different components in the complete system are known or can be estimated, they can be compensated for in the steering-algorithm running on the rapid-prototyping system.

As outliined in section \ref{sec:measurement_setup} the dolly is equipped with a system to determine the deflection angle of the drawbar and both the kingpin angles of the LVC. The sensors' raw signal is then parsed and filtered in an integrated low-level system which feeds the filtered signals to a private CAN-bus, where it is picked up by the MABII. The filtering operation takes a certain time and thus induces a delay.  

Furthermore the model running on the rapid-prototyping system needs a certain time to calculate the current desired steering angle for the dollys' wheels. This has to be determined as well. The steering mechanisms on the dolly are also a delay-inducer due to the inertia in the hydro-mechanic system. This is as well unavoidable, but when measured can as well to a certain degree be compensated for. 

The measuring chain for the logging solution will also be discussed and the different delays will be discussed. 

\subsection{Measured input delay}
\label{sec:measuring_delay}

The response time of the dolly's steering system was determined in workshop environment in stand-still without any additional axle loads. Tests were conducted for both axles individually to ensure that maximum power was available preventing further delays through insufficient electrical power or hydraulic pressure. A repeated steering angle in the form of a step input was used to determine the reaction time of the system. The step amplitude was increased over different tests to account for possible dependencies on articulation of the steering system. The tests each started and ended with a steering angle of $0^\circ .$ 

To eliminate tire friction the investigated axle was suspended in the air in further series of testing by slightly craning up the dolly's front/rear (see figure \ref{fig:dolly_craned_up}) to allow for free movement. An example for the utilized testing routine can be gathered from table \ref{tab:test_matrx_delay_measuring}. Between the different steps (double amplitude) a sufficient wait time of 3500ms was used to allow the system to reach the requested steering angle. This waiting time was kept constant throughout all tests. \\

	\begin{table}[h]
	\label{tab:test_matrx_delay_measuring}
	\centering
	\begin{tabular}{llll}
		\textbf{\#} & \textbf{Amplitude [$  ^\circ $]} & \textbf{Axle}  & \textbf{Comment}                                                                  \\
		1  & 1                & front & lift front axle, ETS dead band \\
		2  & 2                & front &                                                                          \\
		3  & 3                & front &                                                                          \\
		4  & 4                & front &                                                                          \\
		5  & 5                & front &                                                                          \\
		6  & 6                & front &                                                                          \\
		7  & 7                & front &                                                                          \\
		8  & 8                & front &                                                                          \\
		9  & 9                & front & error mode, step too big                                                 \\\hline
		10 & 1                & rear  & lift rear axle, ETS dead band  \\
		11 & 2                & rear &                                                                          \\
		12 & 3                & rear &                                                                          \\
		13 & 4                & rear &                                                                          \\
		14 & 5                & rear &                                                                          \\
		15 & 6                & rear &                                                                          \\
		16 & 7                & rear &                                                                          \\
		17 & 8                & rear &                                                                          \\
		18 & 9                & rear & error mode, step too big                                                                    \\
		             
	\end{tabular}
	\caption{Test matrix for delay measuring with lifted axle (analogous matrix for non-lifted testing)}
\end{table}


\subsection{Delays in logging measuring chain}
\label{sec:delays_on_arduino}


\section{Vehicle testing}
\label{sec:vehicle-testing}

\subsection{System calibration}
\subsection{Actuator tests}
\subsection{Algorithm evaluation}
\subsection{Sensor testing}

\begin{itemize}
	\item test angle sensor offline
	\item zero IMUs, make sure paired sensors show the same outputs
	
\end{itemize}

\section{Hardware-in-the-loop testing}
\begin{itemize}
	\item description of HIL
	\item Functional architecture diagram (MSD replaced with actual dolly)
	\item Limitations (axle loads, standing still)
\end{itemize}

\subsection{Low-speed controller}
\subsection{High-speed controller}

\section{Track testing}
\label{sec:track-testing}

\subsection{Testmaneuvers}

\begin{itemize}
	\item lit research for standard maneuvers
	\item sine-wave
	\item outline critical parts of maneuver
	\item figure with SA over time
	\item expected behaviour from simulation
\end{itemize}

\subsection{Testenvironment AstaZero}

\begin{itemize}
	\item overview of AZ
	\item map in appendix?
	\item restrictions of environment
	
\end{itemize}

\subsection{Testmatrix}

\begin{itemize}
	\item checklist for launch
	\item parameters that very varied
	\item different runs
	\item planned maneuvers
	
\end{itemize}

\subsection{Test setup and instrumentation}

\begin{itemize}
	\item detailed description of placement of sensors, wiring, logging-PC
\end{itemize}


\end{document}
