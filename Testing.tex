% Sample file on how to use subfiles.
\documentclass[ExampleMasters.tex]{subfiles}

\begin{document}
\clearpage
\chapter{Verification and validation of steering  (2-3 Seiten)}
\label{chap:testing}

\section{Overview}
In different states of the project different kind of tests were performed. As a first step bench tests were done to verify the developed software. Following that, several tests on the actual dolly, with the dolly in standstill and no trailers connected, were made.

\section{Bench-Testing}
\label{sec:bench-testing}
\subsection{ECU-setup}
\subsection{VTM maneuver verification}
\begin{itemize}
	\item include simulation results for maneuver of controller being run inside VTM
	\item perhaps include some picture of the video-output, with the actuated truck
	\item plots for maneuver (steering input, speed over time)
	
\end{itemize}
\subsection{CAN verification}
\begin{itemize}
	\item check output from ASF with arduino
	\item check ECU CANs, make sure they differ in output
	
\end{itemize}

\subsection{Fault detection system verification}
\label{sec:fault_detect_test}

\begin{itemize}
	\item send faulty inputs
	\item show plot of faulty input and system reaction
	
\end{itemize}
\section{Processing time evaluation}
\label{chap:processing_time_delay}
\subsection{Background}
The desired solution is supposed to operate at any speed. For high speeds a quick processing and transmission time is required to ensure prompt and realtime intervention of the control system based on the measured input signals. If the delays induced by the different components in the complete system are known or can be estimated, they can be compensated for in the steering-algorithm running on the rapid-prototyping system.

The dolly is equipped with a system to determine the deflection angle of the drawbar. This is measured at the kingpin. The sensor's raw signal is then parsed and filtered in a low-level system which feeds the filtered signals to the CAN-bus, where it is picked up by the MABII. The filtering operation takes a certain time and thus induces a delay. 

Furthermore the model running on the rapid-prototyping system needs a certain time to calculate the current desired steering angle for the dollys' wheels. This has to be determined as well. The steering mechanisms on the dolly are also a delay-inducer due to the inertia in the hydro-mechanic system. This is as well unavoidable, but when measured can as well be compensated for. 


In this chapter the accuracy and delays in the Arduino measuring set-up, see also section \ref{sec:measurement_setup}, is outlined as well. Which is important for determining the hardware-induced delays.


\subsection{Measured input delay}
\label{sec:measuring_delay}
\begin{itemize}
	\item critical transmission time CAN (worst case)
	\item I$^2$C delay
	\item IMU-frequency
	\item msg-frequency from vehicle/dolly
	
\end{itemize}

Two bus-protocols are used to aquire the data measured from the IMU and transmit  to the MABII. The data0 is send via I$^2$C-protocol to the Arduino Due, where it is processed and then send out (to the MABII) on a private CAN-bus. 

\subsection{Computational delay}
\label{sec:computational_delay}
\begin{itemize}
	\item filtering on arduino
	\item averaging
	\item drift correction
	
\end{itemize}

\section{Vehicle testing}
\label{sec:vehicle-testing}

\subsection{System calibration}
\subsection{Actuator tests}
\subsection{Algorithm evaluation}
\subsection{Sensor testing}

\begin{itemize}
	\item test angle sensor offline
	\item zero IMUs, make sure paired sensors show the same outputs
	
\end{itemize}

\section{Hardware-in-the-loop testing}
\begin{itemize}
	\item description of HIL
	\item Functional architecture diagram (MSD replaced with actual dolly)
	\item Limitations (axle loads, standing still)
\end{itemize}

\subsection{Low-speed controller}
\subsection{High-speed controller}

\section{Track testing}
\label{sec:track-testing}

\subsection{Testmaneuvers}

\begin{itemize}
	\item lit research for standard maneuvers
	\item sine-wave
	\item outline critical parts of maneuver
	\item figure with SA over time
	\item expected behaviour from simulation
\end{itemize}

\subsection{Testenvironment AstaZero}

\begin{itemize}
	\item overview of AZ
	\item map in appendix?
	\item restrictions of environment
	
\end{itemize}

\subsection{Testmatrix}

\begin{itemize}
	\item checklist for launch
	\item parameters that very varied
	\item different runs
	\item planned maneuvers
	
\end{itemize}

\subsection{Test setup and instrumentation}

\begin{itemize}
	\item detailed description of placement of sensors, wiring, logging-PC
\end{itemize}


\end{document}
