% Sample file on how to use subfiles.
\documentclass[ExampleMasters.tex]{subfiles}

\begin{document}
\clearpage	
\chapter{Introduction}
\label{chap:introduction}

\section{Purpose}
\label{sec:purpose}
Heavy goods-transport on the road has constantly increased over the last decades. Coupled with the stricter environmental regulations concerning CO\textsubscript{2}-emissions and pollution, the call for more economical transport solution has led to the wider introduction of long combination vehicles. Those truck-trailer combinations have a longer history in areas with low population density, mining and transport within factory sites where rail-road transport is not a viable option but large volumes and tonnages are needed. The prospects of saving costs on driver's salarys, reduced fuel consumption and decreased costs suggests the introduction of those combinations in other areas as well. 

The driving behaviour of LCVs is in many ways different to that of the standard truck and needs to be researched in great detail to gain an understanding of the vehicle's dynamic properties, that is equally detailed as it is for other vehicles class. This will lead to development of better safety systems and thus reduction in accidents and fatalities involving this emerging mode of transportation. 

The major cause for accidents involving LCVs is loss of control in cornering situations.

\section{Limitations}
\label{sec:limitations}
The actual algorithm to compute the desired angle of the dolly's steerable axles is not in the scope of this thesis. Nevertheless to establish an easier insight into the interfaces' parameters, an overview of the structure, in- and outputs of the underlying computational steering model is needed and shall be presented in chapter XXX. 

The hardware- and low-level control-system of the hydraulic actuators is in place already and thus will not be part of this thesis. It is supplied as turn-key software by the manufacturer and readily available on the dolly's electrical control units (ECU). The ECU's software version will be available fully calibrated and parametrized for the dolly at hand and thus provide a reliable working base to build upon. 


\section{Structure of this work}
\label{sec:structure}


\end{document}
