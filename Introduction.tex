% Sample file on how to use subfiles.
\documentclass[ExampleMasters.tex]{subfiles}

\begin{document}
\clearpage
{\pagestyle{empty}\cleardoublepage}%	
\chapter{Introduction (3 Seiten)}
\label{chap:introduction}

\section{Purpose}
\label{sec:purpose}
Heavy goods-transport on the road has constantly increased over the last decades. Coupled with the stricter environmental regulations concerning CO$_{2}$-emissions and pollution, the call for more economical transport solution has led to the wider introduction of \gls{HCT} vehicles. Those truck-trailer combinations have a longer history in geographical areas with low population density, mining and transport within factory sites where rail-road transport is not a viable option but transportation of large volumes and tonnages are called for. The prospects of saving costs on driver's salaries, reduced fuel consumption and decreased costs suggests the introduction of those combinations in other environments as well. Introduction of a new vehicle class leads to many challenges in safety, research and development, and legislation. 

The driving behaviour of \gls{HCT} vehicles is in many ways different to that of standard trucks and needs to be researched in great detail to gain an understanding of the vehicle's dynamic properties, that is equally detailed as it is for other vehicle classes. This will lead to development of better safety and assistance systems and thus reduce threat potential, accidents and fatalities involving this emerging mode of transportation. Different usage patterns of \gls{HCT} vehicles have to be considered as well, when developing functions for \gls{HCT} vehicles. For example inner-city use is no prevalent use-case for \gls{HCT} vehicles, whereas highway safety features and handling properties at higher speeds are prime goals due to high percentage of highway-driving for \gls{HCT} vehicles. Nevertheless maneuverability for docking is also a development goal.

Besides the technical implementation, socio-economic aspects have to be considered. Legislation has to be adjusted to allow for longer vehicle classes, including new certification processes and driver training. Furthermore infrastructure might have to be modified or reviewed to accommodate the needs and dimensions of extended truck combinations.

The research project in which this thesis is embedded aims to develop an active dolly, meaning that steering will be autonomously conducted by the dolly based on the driving situation at hand and various vehicle parameters (e.g. speed, steering wheel angle). Furtheron braking capabilities are to be implemented to act in a similiar fashion as an \gls{ESC} by creating a yaw-moment countering undesired vehicle movements. This counter-stearing will be achieved through wheel-individual brake-application. 

This high-level control algorithm will be executed on a rapid-prototyping system which is linked to and controls the dolly. To supply this connection between the hardware and control-algorithm implemented in the modeling-environment Simulink is the main-task of this thesis.    

\section{Objectives}
\label{sec:objectives}
The main-goals that are supposed to be achieved within this thesis' scope of work are: 

\begin{itemize}
	\item{Develop a software interface for the high-level control algorithm implemented in Simulink to be run on a rapid-prototyping system to control the steering system on an active dolly. }
	\item{Develop the physical hardware interface with the dolly; establish a suitable environment connection for the rapid-prototyping system on-board of the dolly} 
	\item{Develop a measuring solution to determine the processing delays in the sensing system as well as the delays introduced by computation and actuator reaction times. Determine and try to minimize these occuring delays.}  
	\item{Verification of designed systems through different stages of \gls{HIL} tests}
	\item{Develop a safety system that continously monitors the active steering system, prevents malicious inputs and triggers necessary warnings.}
	\item{Develop an interface that allows to contiously determine the system's maximum actuation capabilities depending on the system's current properties (axle load, speed, steering angle, yaw-behaviour). Utilize a vehicle dynamics model to find out the critical boundries for \gls{HCT} vehicles' behaviour.}	
	\item{prepare on-track testing (design road-ready measuring equipment, start-up procedure, repeatable pre-recorded maneuver decription for automated testing)}
	
\end{itemize}



\section{Limitations}
\label{sec:limitations}
To implement the actual high-level algorithm to compute the desired angle for the dolly's steerable axles is not in the scope of this thesis. It was developed in the research project in which this thesis is embedded. Nevertheless to establish an easier insight into the interfaces' parameters, an overview of the structure, in- and outputs of the underlying computational steering model is needed and shall be presented in chapter \ref{chap:steering_model}. 

The hardware- and low-level control-system of the hydraulic actuators is in place already and thus will not be part of this thesis. It is supplied as turn-key software by the manufacturer and readily available on the dolly's \gls{ECU}s. Substantial modifications are necessary to achieve the desired goals, though. The \gls{ECU}'s software version will be available fully calibrated and parametrized for the dolly at hand and thus provide a reliable working base to build upon. 

Braking the dolly's axles individually will not be part of the testing and development scope of this work. Nevertheless the conceptual solution for including this functionalities furtheron will be outlined and a foundation for sending actuation request to these systems will be included in the rapid-protoyping interface block resulting from this thesis. 


\section{Structure of this thesis}
\label{sec:structure}

In the first two sections of this thesis a brief overview of the legal situation concerning \gls{HCT} vehicles for different countries, the current state of the art and ongoing research in the field of \gls{HCT} vehicles shall be presented (chapter \ref{chap:overview}). Furthermore an introduction to the model, that will be run on the rapid-prototyping system will be given (chapter \ref{chap:steering_model}). Those two chapters are meant to give an introduction into the matter and are mainly based on literature review. 

In the succeeding chapters the conducted development work will be described in detail. As a loose guiding structure for this work the development process after the V-model (see \ref{sec:rapid_proto}) will be referred to. Starting with a description of the utilizied hardware-systems and their interconnections in chapter \ref{chap:hardware_setup}, followed by detailing the different software-tools and environments running on those hardware-platforms in chapter \ref{chap:software_setup}. In chapter \ref{chap:processing_time_delay} the measuring concepts and theoretical details for the determination of the overall processing delays in the control-chain will be discussed. As at the planned high speeds and great inertia for testing safety is a major concern, safety functions will be implemented and systematically evaluated. This will be outlined and discussed in chapter \ref{chap:fault_detection}. After the development work was described the created (sub-)systems will subsequently be validated and peu-\`{a}-peu put together and following the V-model evaluated at different stages of integration (chapter \ref{chap:testing}). 

The work closes with a discussion of the results collected during testing and a conclusion where the authors will try to give recommendation for practical implementation and outline future research work in the field.

\end{document}
