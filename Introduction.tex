% Sample file on how to use subfiles.
\documentclass[ExampleMasters.tex]{subfiles}

\begin{document}
\clearpage	
\chapter{Introduction}
\label{chap:introduction}

\section{Purpose}
\label{sec:purpose}
Heavy goods-transport on the road has constantly increased over the last decades. Coupled with the stricter environmental regulations concerning CO\textsubscript{2}-emissions and pollution, the call for more economical transport solution has led to the wider introduction of long combination vehicles. Those truck-trailer combinations have a longer history in areas with low population density, mining and transport within factory sites where rail-road transport is not a viable option but transportation of large volumes and tonnages are called for. The prospects of saving costs on driver's salarys, reduced fuel consumption and decreased costs suggests the introduction of those combinations in other areas as well. Introduction of a new vehicle class leads to many challenges in safety, legislation, and research and development. 

The driving behaviour of LCVs is in many ways different to that of the standard truck and needs to be researched in great detail to gain an understanding of the vehicle's dynamic properties, that is equally detailed as it is for other vehicle classes. This will lead to development of better safety systems and thus reduction in accidents and fatalities involving this emerging mode of transportation. Different usage patterns of LCVs have to be considered as well, when developing functions for LCV. For example inner-city use is no prevalent use-case for LCV, whereas highway safety features and handling properties at higher speeds are prime goals due to high percentage of highway-driving for LCV.

Besides the technical implementation, socio-economic aspects have to be considered. Acceptance of longer 

The research project in which this thesis is embedded aims to develop an active dolly, meaning that steering will be autonomously conducted by the dolly based on the driving situation at hand and various vehicle parameters (e.g. speed, steering wheel angle). Furtheron braking capabilities are to be implemented to act in a similiar fashion as an electronic stability control system (ESC) by creating a yaw-moment countering undesired vehicle movements. This counter-stearing will be achieved through wheel-individual brake-application. 

The high-level control algorithm will be executed on a rapid-prototyping system which is linked to and controls the dolly. To supply this connection between the hardware and control-algorithm implemented in the modelling-environment Simulink is the main-task out this thesis.   

\section{Objectives}
\label{sec:objectives}
The main goals

\section{Limitations}
\label{sec:limitations}
The actual algorithm to compute the desired angle of the dolly's steerable axles is not in the scope of this thesis. Nevertheless to establish an easier insight into the interfaces' parameters, an overview of the structure, in- and outputs of the underlying computational steering model is needed and shall be presented in chapter XXX. 

The hardware- and low-level control-system of the hydraulic actuators is in place already and thus will not be part of this thesis. It is supplied as turn-key software by the manufacturer and readily available on the dolly's electrical control units (ECU). The ECU's software version will be available fully calibrated and parametrized for the dolly at hand and thus provide a reliable working base to build upon. 


\section{Structure of this work}
\label{sec:structure}

In the first two sections of this thesis a brief overview of the legal situation for different countries and the current state of the art and ongoing research shall be presented (chapter \ref{chap:overview}). Furthermore an introduction to the model, that will be run on the rapid-prototyping system will be given (chapter \ref{chap:steering_model}). They are meant to give an introduction into the matter. In the succeeding chapters the conducted work will be described in detail. Starting with a description of the utilizied hardware-systems and its interconnections in chapter \ref{chap:hardware_setup}, followed by detailing the different software-tools and environments running on those platforms in chapter \ref{chap:software_setup}.

\end{document}
