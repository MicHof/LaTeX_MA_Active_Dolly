% Sample file on how to use subfiles.
\documentclass[ExampleMasters.tex]{subfiles}

\begin{document}
\clearpage


\chapter{Fault detection and system ability}
\label{chap:fault_detection}
\section{Failure Mode and Effects Analysis (FMEA)}
\label{sec:FMEA}
FMEAs were first used by NASA in the apollo project in the 1960s.
A FMEA is used to detect possible failures before they appear. Therefore the FMEA is done in a early stage of a project in order to be able to take the results of the FMEA into consideration when developing a system.
In a first step a block diagram of the system with all of it's inputs, outputs and subsytems was created to gain a complete understanding of the system. Every subsystem was broken down to the lowest level and for those subsystems block diagrams with all the components, inputs and outputs were created as well.

Starting from the block diagrams all potential failure modes were determined for every component of the system respectively subsystem. As a next step the potential effects of these failure modes were determined and the severity of these effects were evaluated on a scale from one to ten,  with 1 being the lowest severity and ten being the highest severity. For the evaluation a table that shows how different severities correlate with the numbers was used. \colorbox{yellow}{picture of severity-table}    


\begin{itemize}
	\item system discription $\rightarrow\;$ diagram of system and subsystems
	\item break system and subsystems down into components and in/out-puts
	\item find for each component all:
	\begin{itemize}
		\item potential failure modes
		\item potential effects of these failures
	\end{itemize}
	\item define severity of each effect (scale from 0-10)
	\item find potential causes for each failure mode
	\item define probability of each cause
	\item list current design controls, that help to detect the specific failure
	\item define detectability of each failure
	\item calculate risk priority number (rpn=sev*prop*detec)
	\begin{itemize}
		\item rpn<40 $\rightarrow\;$ no action needed
		\item 40<rpn<100 $\rightarrow\;$ take closer look
		\item rpn>100 $\rightarrow\;$ action needed
	\end{itemize}
	\item for failures, where action is needed--> recommend actions that prevent failures
	\item define new sev., prop. and detect.
	\item calculate new rpn
	\item if needed $\rightarrow\;$ iteration of process    
\end{itemize}
\section{Safety concepts}
\label{sec:safetyconcepts}

\section{Maximum capabilities of the system}
\label{sec:maxi_capabilities}
\begin{itemize}
	\item give indicator of maximum angle/angle rate
	\item describe "algorithm"/lookuptable
	\item explain underlying physical correlation (ref to MA from )
\end{itemize}

\section{Warning and state-info system}
\label{sec:warning_system}
\begin{itemize}
	\item warnings from dolly ECU
	\item warnings from EBS
	\item warnings from vehicle
	\item 'own' error codes and warnings (e.g. logging, MABII related, arduino-IMU related)
\end{itemize}




\end{document}
