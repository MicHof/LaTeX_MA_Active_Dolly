% Sample file on how to use subfiles.
\documentclass[ExampleMasters.tex]{subfiles}

\begin{document}
\clearpage


\chapter{Fault detection and system ability}
\label{chap:fault_detection}
\section{Failure Mode and Effects Analysis (FMEA)}
\label{sec:FMEA}
\begin{itemize}
	\item system discription $\rightarrow\;$ diagram of system and subsystems
	\item break system and subsystems down into components and in/out-puts
	\item find for each component all:
	\begin{itemize}
		\item potential failure modes
		\item potential effects of these failures
	\end{itemize}
	\item define severity of each effect (scale from 0-10)
	\item find potential causes for each failure mode
	\item define probability of each cause
	\item list current design controls, that help to detect the specific failure
	\item define detectability of each failure
	\item calculate risk priority number (rpn=sev*prop*detec)
	\begin{itemize}
		\item rpn<40 $\rightarrow\;$ no action needed
		\item 40<rpn<100 $\rightarrow\;$ take closer look
		\item rpn>100 $\rightarrow\;$ action needed
	\end{itemize}
	\item for failures, where action is needed--> recommend actions that prevent failures
	\item define new sev., prop. and detect.
	\item calculate new rpn
	\item if needed $\rightarrow\;$ iteration of process    
\end{itemize}
\section{Safety concepts}
\label{sec:safetyconcepts}

\section{Maximum capabilities of the system}
\label{sec:maxi_capabilities}
\begin{itemize}
	\item give indicator of maximum angle/angle rate
	\item describe "algorithm"/lookuptable
	\item explain underlying physical correlation (ref to MA from )
\end{itemize}

\section{Warning and state-info system}
\label{sec:warning_system}
\begin{itemize}
	\item warnings from dolly ECU
	\item warnings from EBS
	\item warnings from vehicle
	\item 'own' error codes and warnings (e.g. logging, MABII related, arduino-IMU related)
\end{itemize}




\end{document}
