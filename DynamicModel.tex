% Sample file on how to use subfiles.
\documentclass[ExampleMasters.tex]{subfiles}

\begin{document}
\clearpage
\chapter{Steering Model}
\label{chap:steering_model}

\section{Rapid Control Prototyping}
\label{sec:rapid_proto}

In automotive software development standardized processes are applied to assure structured work. The V-model constitutes one graphic representations of the chronological sequence of the steps towards the finished software product. It also gives an approximate impression about the level of detail of each of the steps. A simplified version of this V-model can be seen in figure . A similar sequence of development steps is applied to develop the hardware platform in parallel.

The complexity of mechatronical systems lead to new development methods that made it possible to take 'short-cuts' in the established V-model process. In utilizing these new methods such as rapid control prototyping and hardware-in-the-loop testing, a tremendous decrease in development time can be achieved. For the research project in which this theses is embedded it even opens up the possibility of on-road testing, as it eliminates the low-level development steps of the actual implementation and some of the detailed design work.


\begin{itemize}
	\item V-model
	\item real-time capability
	\item abstract high level model to programm ==> focus on function development and modelling, no low level coding needed
	\item time critical processes
	\item extensive lecture notes from \cite{rapidcontrolprototyping}
\end{itemize}

\section{Overview of the model}
\label{sec:overview_of_the_model}
			
\begin{itemize}
	\item based on lit from MI-paper 
	\item overview graph of model
	\item single track model
	
	\item filtering of inputs?
	\item feedback-loop
	\item start condition
	\item what can be concluded with the model?
\end{itemize}

\section{Input parameters}
\label{sec:input_parameters}

\begin{itemize}
	\item what parameters are used as inputs? dimensions, min/max
	\item explain feedback/inverse path
	
\end{itemize}

\section{Real-Time implementation}
\label{sec:real_time_implementation}

\begin{itemize}
	\item what had to be changed to allow for MABII execution?
	\item how will feedback loop be handled?
	item incorporate measurings?
	\item simulation step size?
	\item utilized computational method
\end{itemize}

\section{Interface with Real-Time environment}
\label{sec:interface_with_real_time}

\begin{itemize}
	\item capabilities of system (see section \ref{sec:maxi_capabilities})
	\item signals passed forward from environment + restriction
	
	\item actuator signals
	\item signal modification (correct frequency for CAN)
	\item safety flags/signals
	
\end{itemize}

\end{document}
