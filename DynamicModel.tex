% Sample file on how to use subfiles.
\documentclass[ExampleMasters.tex]{subfiles}

\begin{document}
\clearpage
\chapter{Steering Model}
\label{chap:steering_model}

\section{Rapid Control Prototyping}
\label{sec:rapid_proto}

This section should give a quite detailed overview over the processes for rapid prototpying of software functions.
\begin{itemize}
	\item V-model
	\item real-time capability
	\item abstract high level model to programm ==> focus on function development and modelling, no low level coding needed
	\item time critical processes
	\item extensive lecture notes from \cite{rapidcontrolprototyping}
\end{itemize}

\section{Overview of the model}
\label{sec:overview_of_the_model}
			
\begin{itemize}
	\item based on lit from MI-paper 
	\item overview graph of model
	\item single track model
	
	\item filtering of inputs?
	\item feedback-loop
	\item start condition
	\item what can be concluded with the model?
\end{itemize}

\section{Input parameters}
\label{sec:input_parameters}

\begin{itemize}
	\item what parameters are used as inputs? dimensions, min/max
	\item explain feedback/inverse path
	
\end{itemize}

\section{Real-Time implementation}
\label{sec:real_time_implementation}

\begin{itemize}
	\item what had to be changed to allow for MABII execution?
	\item how will feedback loop be handled?
	item incorporate measurings?
	\item simulation step size?
	\item utilized computational method
\end{itemize}

\section{Interface with Real-Time environment}
\label{sec:interface_with_real_time}

\begin{itemize}
	\item capabilities of system (see section \ref{sec:maxi_capabilities})
	\item signals passed forward from environment + restriction
	
	\item actuator signals
	\item signal modification (correct frequency for CAN)
	\item safety flags/signals
	
\end{itemize}

\end{document}
