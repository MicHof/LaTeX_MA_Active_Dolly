\begin{document}
\clearpage
{\pagestyle{empty}\cleardoublepage}%

This thesis was carried out at the Vehicle Engineering and Autonomous Systems division at Chalmers. We are very thankful for the friendly and welcoming environment that we found here. Our sincere thanks especially go to Manjurul Islam, PhD � our supervisor � and Prof. Bengt Jacobson who co-supervised us. Immense trust was put in our work in advance and we were provided with the possibility to work independently early on with very advanced technology and had the possibility to develop our concepts with great support and funding within the Active Dolly Build-Up research project. Thank you very much for a very interesting and educational time. \\
As this practical thesis ties in with many suppliers and different parties, we also want to express our gratitude towards the many contributors and interesting discussion partners. First and foremost Leo Laine (Volvo Trucks AB) has to be mentioned, who advised on our thesis and gave some vital input and impulses for our work and linked us with many contacts within Volvo Trucks. Oskar Gellerbrant and Martin H�rberg from Fenco AB, who gave great help and input on dSpace-related issues. Richard Dahlsj� from VM-trailers for providing us with some practical advice and supplies for early prototypes. VSE�s Rik de Zaaijer provided with very detailed information and good input, which greatly accelerated work in the early stages. The local WABCO subsidiary (Joakim Jonsson, Erik Helldin) provided us with nice contacts, diagnoses equipment, great advice and technical help concerning the brake systems on the dolly. The Chalmers workshop provided us with an uncomplicated way to build our lab-prototypes. For further equipment manufacturing we had the nice chance to discuss and cooperate with Segula AB (Jesper Larsson, Arpit Karsolia). 


\end{document}
