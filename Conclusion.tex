% Sample file on how to use subfiles.
\documentclass[ExampleMasters.tex]{subfiles}

\begin{document}
\clearpage
{\pagestyle{empty}\cleardoublepage}%
\chapter{Conclusion (2 Seiten)}
\label{chap:conclusion}

\section{Recommendation}
\label{sec:recommendation}



\section{Future Work}
\label{sec:future_work}

The braking solution that was outlined in this thesis has to be implemented in detail and evaluated with great  precautions before taking it to the track. The safest solution would be to request torques/brake pressures via the brake \gls{ECU}'s diagnosis interface, as this would leave the original braking CAN-communication intact and prevent a solution similiar to the one implemented for the interception and modification of the \gls{ETS} sensor signals. 

An air-suspended leveling system is available on the dolly and \gls{CAN} be controlled fairly easily over the standardized ISO-11992 CAN, physically available on the 7-pin trailer connector which carries the \gls{ABS}/\gls{EBS} signals. It has to be evaluated whether this functionality is available while the combination is moving and of course whether it contributes anything to the enhanced dynamic behaviour of the whole combination.

Wheel individual propulsion via hub-mounted electric/hydraulic motors is already available for cars and leads to a safety system similiar to the working principal of \gls{ESC} but without the disadvantage of slowing down the vehicle. These implementations are usually called torque vectoring or active yaw and greatly increase the performance in highly dynamic situations. Of course it is not possible to drive the second semi-trailer independently for extended periods of time as this exceed the output of the trucks alternator and thus require impossibly large energy-storage. Still besides the great advantages that can be expected in dynamic situation, a electronically propelled dolly would allow independent shunting of the decoupled second semi-trailer unit.



\begin{itemize}
	\item braking (indivi)
	\item leveling
	\item wireless configuration?
	\item track testing
	\item propulsion
	\item 
\end{itemize}

\end{document}
