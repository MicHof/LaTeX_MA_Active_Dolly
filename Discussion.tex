% Sample file on how to use subfiles.
\documentclass[ExampleMasters.tex]{subfiles}

\begin{document}
\clearpage
{\pagestyle{empty}\cleardoublepage}%
\chapter{Discussion}
\label{chap:discussion}

%\section{Results from bench testing}
%\label{sec:results_bench}

%\begin{itemize}%
%	\item lessons-learned?
%	\item adaptation for future projects
%	\item what was taken over for further tests? 
%	\item what couldnt be simultaed?
%\end{itemize}

\section{Results from processing time evaluation}
\label{sec:results_processing_time}

Figure \ref{fig:step_input} shows the requested angle, target angle\footnote{The target angle is an internal parameter within the legacy \gls{VSE} \gls{ETS}. It corresponds to the steering angle that the \gls{ETS}-\gls{ECU} commands to the hydraulic steering. It therefore includes some of the internal limiters and safety systems of the \gls{ETS} and thus moves slower that the requested angle send from this thesis' interface.} and actual angle of the dolly's front axle over time. As described in section \ref{sec:measuring_delay} the requested angle is a step input, in this case of 5$^\circ$.\\
\begin{figure*}[!hbt]
	\centering
	\includegraphics[width=1\linewidth]{figures/lifted_front_5deg}
	\caption{Steering angle request step input 5$^{\circ}$}
	
	\label{fig:step_input}
\end{figure*}
The target angle's slope is constant and lower than the slope of the request angle. The actual angle has approximately the same slope as the target angle, but it is delayed. The actual angle tends to overshoot the requested angle for falling slopes. A reason for this could not be determined. Furthermore the slope of the actual angle is reduced when the actual angle almost reached the request. There is also a saddle point around an angle of 0$^{\circ}$ for rising slopes of the actual angle.          
The target angle's slope is constant and lower than the slope of the request angle. The actual angle has approximately the same slope as the target angle, but it is delayed. The actual angle tends to overshoot the requested angle for falling slopes. A reason for this could not be determined. Furthermore the slope of the actual angle is reduced when the actual angle almost reached the request. There is also a saddle point around an angle of 0$^{\circ}$ for rising slopes of the actual angle.          
The delays over the different request angle step inputs were plotted and a linear regression was made to deduct the actual delay for small changes of the request angle. This delay is the relevant delay, since in real driving situations the requested angle is never a step input but rather changes only slowly. This delay over the step inputs together with the linear regression is shown in figure \ref{fig:delay_testing_lin_interpolation}.    

\begin{figure*}[!hbt]
	\centering
	\includegraphics[width=1\linewidth]{figures/delay_testing_lin_interpolation}
	\caption{Rise time from send out request to achieve a certain steering angle on the dolly.}
	\label{fig:delay_testing_lin_interpolation}
\end{figure*}
The deducted delay is 0.26s for the front axle and 0.3s for the rear axle.\\
During plot analysis of preliminary results it was discussed, whether it would be possible to by-pass the occuring delay by feeding a ramp-input instead of a step-input into the \gls{ETS}. It was assumed that there is a low-pass filter in place which would not affect ramp inputs. Subsequent tests with ramp inputs with varying slopes and amplitudes showed that it was indeed possible to eliminate the low-pass filter leading to no delay between the target angle and the request angle. Examples of the ramp inputs with different rates and the subsequent reaction of the \gls{ETS} system can be seen in the figures \ref{fig:rate_limiter1}, \ref{fig:rate_limiter2} and \ref{fig:rate_limiter3}.\\
\begin{figure*}[!htb]
	\centering
	\includegraphics[width=1\linewidth]{figures/rate_limiter1}
	\caption{Steering angle request ramp input with a rate of 0.01}
	
	\label{fig:rate_limiter1}
\end{figure*}
\begin{figure*}[!hbt]
	\centering
	\includegraphics[width=1\linewidth]{figures/rate_limiter2}
	\caption{Steering angle request ramp input with a rate of 0.001}
	
	\label{fig:rate_limiter2}
\end{figure*}
\begin{figure*}[!hbt]
	\centering
	\includegraphics[width=1\linewidth]{figures/rate_limiter3}
	\caption{Steering angle request ramp input with a rate of 0.1}
	
	\label{fig:rate_limiter3}
\end{figure*}

With a rate of 0.01 the request angle and target angle align. They also align at a rate of 0.001 but for that case their slope is much lower, so it takes an insufficiently long time to reach the indented angle. For a rate of 0.1 the signals of the request angle and the target angle do not align anymore. For rates higher than 0.3, the \gls{ETS}-system switches to alarm mode, because the request angle changes to fast, so the signal is implausible for the system. This corresponds to the behaviour which was caused by sending step inputs (also see section \ref{sec:measuring_delay}).


Comparing figure \ref{fig:rate_limiter1} and figure \ref{fig:rate_limiter3} indicates that it is not possible to get any faster change with a ramp input than with the request of a step input. It thus can be concluded that the maximum steering rate of $5^\circ $/s can not be circumvented.

The processing time evaluation was gathered over 2000 loop cycles. The minimal and maximal execution time can be gathered from table \ref{tab:execution_time}. As can be seen, the deviation in execution time is very small (+- 1$\mu$s). These results present the values gathered after streamlining the original libraries and eliminating many $delay()$-calls which were present to make sure the \gls{CAN}-transmission did not get jammed. After extensive testing it was possible to cut them down to a minimum. 

\begin{table}[h]
	\centering
	\caption{Overview of processing time of different execution blocks in the \acrshort{IMU} measuring setup}
	\label{tab:execution_time}
	\begin{tabular}{l|llllll}
		& read\_gyro & read\_accel & read\_compass & can\_send & serial\_send & sum   \\ \hline 
		minimum ($\mu$s) & 2717       & 864         & 860           & 900       & 4804         & 10146 \\
		maximum ($\mu$s) & 2717       & 865         & 861           & 901       & 4804         & 10147
	\end{tabular}
\end{table}

So the overall execution time of the looped function on the \gls{IMU}-system is 10.1ms, which was sufficiently low to introduce a \gls{GPS}-unit to time-stamp the measuring data. The reading of the \gls{GPS}-unit is also conducted via UART (and thus rather slow). The reading of the \gls{GPS}-unit can be reliably achieved in 3ms; this timespan was also minimized in an iterative fashion. The additional delay introduced by the broadcast of additional \gls{CAN}-messages can be neglected. So the the overall sum for all readings and the subsequent transmission of the data via \gls{CAN} takes 13.1ms. It was decided to introduce a sampling frequency of 50Hz for the measurings to have a round and common number as a sampling frequency.  
The original concern, that the different systems would not run sufficiently fast, which lead to the investigation carried out for this section, can now be dismissed.


%\section{Results from in vehicle testing}
%\label{sec:results_vehicle_testing}




%\begin{itemize}
%	\item CAN-analysis
%%	\item robustness?
%	\item reliability of safety features
%\end{itemize}


\section{Results from hardware-in-the-loop testing}

Figure \ref{fig:HIL002_front} shows the measurements of the ETS-CAN of the front axle of the dolly for the low-speed maneuver described in section \ref{sec:HIL}.\\

\begin{figure*}[!htb]
	\centering
	\includegraphics[width=1\linewidth]{figures/HIL002_front}
	\caption{\acrlong{HIL}-test low-speed front axle}
	
	\label{fig:HIL002_front}
\end{figure*}

The request angle is the angle that was requested from the controller and then sent from the Simulation-PC to the \gls{MABII}. The target angle is the angle that was calculated by the \gls{ETS}-\gls{ECU} and sent as a request to the actuators of the dolly. The actual angle is the angle that the sensors on the dolly's steering knuckles measured and which represent the articulation of the dolly's wheel on the ground.
Figure \ref{fig:HIL002_rear} shows the same measurements for the rear axle.

\begin{figure*}[!htb]
	\centering
	\includegraphics[width=1\linewidth]{figures/HIL002_rear}
	\caption{\acrlong{HIL}-test low-speed rear axle}
	
	\label{fig:HIL002_rear}
\end{figure*}
Since the requested angle for front and rear is the same, below only the front axle is pictured. Both axles show a very similar behaviour and thus only considering one is justifiable.
Figure \ref{fig:HIL002_front_closeup} shows an extraction of the measurements of the front axle. 
\begin{figure*}[!htb]
	\centering
	\includegraphics[width=1\linewidth]{figures/HIL002_front_closeup}
	\caption{Details of \acrlong{HIL}-test low-speed front axle}
	
	\label{fig:HIL002_front_closeup}
\end{figure*} \\
As already mentioned in section \ref{sec:measuring_delay} there is a delay between the angle requested of the \gls{ETS}-\gls{ECU} and the actual angle of the wheels. This can very distinctly be seen in this plot. In comparison to this, the delay between the angle request that is sent from the \gls{MABII} to the \gls{CAN} and the target angle that is sent from the \gls{ETS}-\gls{ECU} to the actuators is very small and can therefore be neglected.\\

To get an impression of the overall results of the \gls{HIL}-test, figure \ref{fig:HIL002_complete} shows the measurements conducted on the simulation-PC and the measurements done in ControllDesk in the same plot. 
\begin{figure*}[!htb]
	\centering
	\includegraphics[width=1\linewidth]{figures/HIL006_alles}
	\caption{\acrlong{HIL}-test front axle measurements from VTM and ETS-CAN}	
	\label{fig:HIL002_complete}
\end{figure*}

For one thing the plot shows the limiting of the angle that is requested from the controller by the supervisor block described in section \ref{sec:warning_system}. For another thing, it shows that there is a relatively high delay between the signal of the actual angle of the dolly from the \gls{ETS}-\gls{CAN} and the signal that is fed-back to the Simulation-PC. This is most likely caused by the use of the serial interface to transmit the signal from the CAN to the simulation-PC. \\
To get a more detailed view of this delay as well as the limitation of the request angle figure \ref{fig:HIL002_complete_details} shows a extraction of figure \ref{fig:HIL002_complete}.   

\begin{figure*}[!htb]
	\centering
	\includegraphics[width=1\linewidth]{figures/HIL006_alles_details}
	\caption{Details of \acrlong{HIL}-test front axle measurements from \acrshort{VTM} and \acrshort{ETS}-\acrshort{CAN}}
	
	\label{fig:HIL002_complete_details}
\end{figure*}


The tested controller was evaluated in \gls{HIL} testing as well as in pure simulation in \gls{VTM}. The two resulting curves of the vehicles path can be seen in figures \ref{fig:xy_VTM} for the pure \gls{VTM}-simulation and \ref{fig:xy_HIL} for the \gls{HIL}-testing of the same maneuver respectively. For easier comparison figure \ref{fig:xy_HIL_and_VTM} depicts the driven paths for both environments. The steering reacts a bit slower for the \gls{HIL}-tests, but overall the paths are very similar. The actual performance of the algorithm and its evaluation is not part of this thesis; \gls{HIL}-testing was done to validate the matching between simulation and \gls{VTM} environment and also to proof the functioning of the developed interface.


\begin{figure*}[!htb]
	\centering
	\includegraphics[width=1\linewidth]{figures/xy_HIL_and_VTM}

	\caption{Comparison between swept paths of different units in the \acrshort{HCT} combination during \acrlong{HIL}-testing and \acrshort{VTM}-simulation}
	
	\label{fig:xy_HIL_and_VTM}
\end{figure*}

As can be gathered from figure \ref{fig:compare_HIL_VTS}, the error between the two environments accumulates over time. This is a consequence of slightly different paths they choose due to the dolly not really allowing steering of very small angles (-2 to 2$^\circ$). This is caused by hardware limitations and a dead-band filter which is present in the \gls{ETS}-\gls{ECU}. The end of the measuring can practically be ignored from approximately 100m of distance traveled, as this is mostly caused by different angles at which the combination exits the turn. This can be compensated for by fine-tuning the steering inputs, which was omitted for this thesis, as it was only the aim to demonstrate the general function of the controller. Fine-tuning will only be conducted once the final controller is ready and correctly implemented. 

\begin{figure*}[!htb]
	\centering
	\includegraphics[width=1\linewidth]{figures/compare_HIL_VTS}
	\caption{Deviation between \acrlong{HIL}-testing and \acrshort{VTM}-simulation plotted over distance traveled during 180$^\circ$-turn}
	
	\label{fig:compare_HIL_VTS}
\end{figure*}


\end{document}
