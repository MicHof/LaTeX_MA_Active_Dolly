% Sample file on how to use subfiles.
\documentclass[ExampleMasters.tex]{subfiles}

\begin{document}
\clearpage
\chapter{Discussion (3-4 Seiten)}
\label{chap:discussion}

\section{Results from bench testing}
\label{sec:results_bench}

\begin{itemize}
	\item lessons-learned?
	\item adaptation for future projects
	\item what was taken over for further tests? 
	\item what couldnt be simultaed?
\end{itemize}

\section{Results from processing time evaluation}
\label{sec:results_processing_time}

\begin{itemize}
	\item diagrams (over amplitude, show time)
	\item explain different curves
	\item explain data gathering
	\item mention ramp input 
	\item show reaction time
\end{itemize}

During plot analysis of preliminary results it was discussed, whether it would be possible to by-pass the occuring delay by feeding a ramp-input instead of a step-input into the \gls{ETS}. It was assumed that there is a low-pass filter in place which would not affect ramp inputs. Subsequent tests with ramp inputs with varying slopes and amplitudes showed that it was indeed possible to eliminate the low-pass filter leading to no delay between the target angle and the request angle. Nevertheless it was not possible to get any faster change than with the request of a step input. It thus can be concluded that the maximum steering rate of $5 ^\circ /s$ can not be circumvented.

\section{Results from in vehicle testing}
\label{sec:results_vehicle_testing}

\begin{itemize}
	\item CAN-analysis
	\item robustness?
	\item reliability of safety features
\end{itemize}
\section{Results from hardware-in-the-loop testing}
\begin{itemize}
	\item compare plots from \gls{HIL}  with simulation
\end{itemize}

\section{Comparison}
\label{sec:results_comparrison}
\begin{itemize}
	\item \gls{VTM}  $<=>$ testing
\end{itemize}







\end{document}
